\documentclass[../Language-declaration.tex]{subfiles}

\begin{document}
    \chapter{Цель создания языка}

        \section{Какую проблему решает}
            \subsection{Со стороны пользователя}
                Иногда для использования программы на каком то моменте хочется в место константного параметра указать формулу для вычисления которая может взаимодействовать с программой. Как 1 из вариантов это изменение размера 3D элемента в сцене по формуле вида "$abs(\sin($frame$*pi))*45$"\footnote{Это всего лишь пример.}.

            \subsection{Со стороны разработчика}
                Удобный интерфейс для решения проблемы пользователя, в прочем, как и любая другая библиотека.


        \section{Итоговое видение для пользователя}

            \subsection{Кратко}
                Библиотека для C++ которая парсит строку и выполняет те выражение, которые записаны в ней. То есть, он принимает на вход строку (формулу) и выполняет её.

            \subsection{Полно}
                Библиотека которая даёт возможность указать пользователю не только числовое значение (как в обычных программах, к пример "Количество точек: 5", а указать формулу, которая может взаимодействовать с окружением "Количество точек: rand() \% 100 * frame()"\footnote{То есть, взять остаток от деления случайного числа на 100 и умножить на номер текущего кадра.})

            \subsection{Более понятным языком}
                Это по сути формульный язык. Применить его можно тогда, когда нужно у пользователя уточнить не константное выражение, а выражение которое можно вычислить. На пример в программе для анимации можно будет указать формулу вида "sin(frame)" и получить на выходе число, которое каждый кадр будет браться по этой формуле.
\end{document}
