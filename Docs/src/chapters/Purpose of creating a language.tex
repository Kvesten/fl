\documentclass[../Language-declaration.tex]{subfiles}

\begin{document}
    \chapter{Цель создания языка}

        \section{Какую проблему решает}
            \subsection{Со стороны пользователя}
                Иногда для использования программы на каком то моменте хочется в место константного параметра указать формулу для вычисления которая может взаимодействовать с программой. Как 1 из вариантов это изменение размера 3D элемента в сцене по формуле вида "$abs(\sin($frame$*\pi))*45$"\footnote{Это всего лишь пример.}. То есть она принимает из окружения номер текущего кадра, умножает его на $\pi$ и от этого мракобесия берёт $\sin$ и умножает его на 45.\footnote{Зачем? Да я хз...}

            \subsection{Со стороны разработчика}
                Удобный интерфейс для решения проблемы пользователя, в прочем, как и любая другая библиотека.
\end{document}
