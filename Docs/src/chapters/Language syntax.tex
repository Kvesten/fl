\documentclass[../Language-declaration.tex]{subfiles}

\begin{document}
    \chapter{Синтаксис языка}
        \section{Типы данных}
            Так как FL - формульный язык, в переданной интерпретатору формуле нельзя принудительно указывать типы данных (да и, впрочем, не имеет смысла), но при парсинге интерпретатор считывает лексемы по типам. Всего формальных типов три:
            \begin{itemize}
             \item литеральный bool - булевый тип, гарантируется, что он хранит числа в диапазоне [0,1]. Все числа в этом диапазоне изначально считают булевым. Пример: "1";
             \item литеральный integer - целый тип. Пример: "100";
             \item литеральный float - шаблонный тип данных с плавающей запятой. Пример: "100.23e-2";
             \item вызов функции: если литералы пишутся пользователем непосредственно в коде, то функции поставляются в конструктор объекта интерпретатора и вызываются во время исполнения вашей программы на С++. Пример: "rand()".
             \item переменная из С++: переменная может иметь любой тип
            \end{itemize}

            Особенности взаимодействия объектов от разных типов данных необходимо учитывать, чтобы избежать возможных ошибок при составлении формулы, которая отправится на корм интерпретатору.

        \section{Операторы}
            Интерпретатор поддерживает следующие операторы:
        \begin{enumerate}
         \item + - оператор сложения;
         \item $-$ - оператор вычитания;
         \item * - оператор умножения;
         \item \% - оператор взятия по модулю;
         \item / - оператор деления. Результат оператора всегда возвращает float;
         \item Булевые операции:
            \subitem \& - булевое "и";
            \subitem | - булевое "или";
            \subitem ! - булевое "не";
        \end{enumerate}

           Тип float имеет преимущество над int: любые операции, параметрами которых являются литералы (функции) от типов int и float одновременно, возвращают float.\newline

           Булевые операции всегда возвращают тип bool. Примеры:
           \begin{enumerate}
            \item "187.3 | 120" вернёт "1";
            \item "45 \& 0" вернёт "0".
           \end{enumerate}





\end{document}
