\documentclass[../Language-declaration.tex]{subfiles}

\begin{document}
    \chapter{Синтаксис языка}
        \section{Операторы}
            Интерпретатор поддерживает следующие операторы:
        \begin{enumerate}
         \item + - оператор сложения;
         \item $-$ - оператор вычитания;
         \item * - оператор умножения;
         \item \% - оператор взятия по модулю;
         \item / - оператор деления. Результат оператора всегда возвращает float;
         \item Булевые операции:
            \subitem \& - булевое "и". Для удобства поддерживается зарезервированное слово "and";
            \subitem | - булевое "или". Для удобства поддерживается зарезервированное слово "or";
            \subitem ! - булевое "не". Для удобства поддерживается зарезервированное слово "not";
            \subitem  - булевое "исключащее или".  Для удобства поддерживается зарезервированное слово "xor".
        \end{enumerate}
        Таким образом, при использовании в качестве шаблона не примитивный тип, а класс, у класса должны быть определены данные операторы.

        \section{Использование функций и переменных среды}
            При создании шаблонного объекта интерпретатора в конструктор помещаются два объекта стандартного класса map: map<имя функции типа std::string, struct { указатель на функцию, количество параметров функции типа шаблона }> и map<имя переменной типа std::string, значение переменной>. Все переменные и значения, возвращаемые функциями, должны иметь шаблонный тип.

            Вызов переменной происходит простейшим образом: в формуле пишется её имя.
            Пример: "frame - 120 * 780".

            Вызов функции отличается только наличием круглых скобок после названия функции вместе с заранее определенным количеством параметров.
            Пример: "sin(x) / 2 + func(1240, sin(x))".

            Здесь "func" - функция, принимающая в себя два параметра.

            Таким образом, в формуле можно использовать переменные и функции, имеющие одинаковые названия, так как интерпретатор отличает функцию от переменной наличием скобок.

\end{document}
